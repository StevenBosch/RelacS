The back-end will first convert the input to a cochleogram. This has not yet been implemented in the final product.

To extract stressful sounds from the cochleograms, two approaches are used. For the first approach the cochleogram is sliced into windows of about $\frac{2}{3}$ seconds that overlap for $50\%$. The probability that such a window contains a stressful sound is determined by looking solely at the mean intensity values per frequency band using a naive Bayes classifier.  

The second approach consist mainly of a convolutional neural network (CNN). For this approach, first, the cochleogram is preprocessed so that sounds that have a shorter duration than 1 second are surpressed. Windows are extracted similarly. From these windows, features are extracted by the CNN. These are used as input for a multilayer perceptron, that outputs a the probability that the window is stressful. 

These methods are combined into an array of stress level probabilities. Parts of sounds that have a very large amplitude with respect to the rest of a window are also marked stressful. The output of the back-end is thus an array of stress levels per window.



%When the user makes a recording it is send to the back-end for processing. 
%The back-end will first convert the input to a HDF5
%file and do all the related processing. After this is done it will apply several
%machine learning methods. The results of all the machine learning methods are
%combined and a histogram of the stress levels in the file will be created based
%on the stress levels of each window in the input file.
